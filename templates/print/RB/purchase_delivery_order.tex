\input{inheaders.tex}


% Variablen, die in settings verwendet werden
\newcommand{\lxlangcode} {<%template_meta.language.template_code%>}
\newcommand{\lxmedia} {<%media%>}
\newcommand{\lxcurrency} {<%currency%>}
\newcommand{\lxtitlebar} {<%titlebar%>}

% settings: Einstellungen, Logo, Briefpapier, Kopfzeile, Fusszeile
\input{insettings.tex}


% laufende Kopfzeile:
\ourhead{}{}{\einkaufslieferschein}{<%donumber%>}{<%dodate%>}


\begin{document}

\ourfont

\begin{minipage}{8cm}
  <%name%>

  <%street%>

  <%zipcode%> <%city%>

  <%country%>
\end{minipage}
\hfill
\begin{minipage}{6cm}
  \rightline{\LARGE\textbf{\textit{\einkaufslieferschein}}} \vspace*{0.2cm}
  \rightline{\large\textbf{\textit{\nr ~<%donumber%>
  }}} \vspace*{0.2cm}

  \datum:\hfill <%dodate%>

  <%if cusordnumber%>\unsereBestellnummer:\hfill <%cusordnumber%><%end if%>

  <%if ordnumber%>\auftragsnummer:\hfill <%ordnumber%><%end if%>

  \ansprechpartner:\hfill <%employee_name%>

  <%if globalprojectnumber%> \projektnummer:\hfill <%globalprojectnumber%> <%end globalprojectnumber%>
\end{minipage}

\vspace*{1.5cm}


%
% - longtable kann innerhalb der Tabelle umbrechen
% - da der Umbruch nicht von Lx-Office kontrolliert wird, kann man keinen
%   Übertrag machen
% - Innerhalb des Langtextes <%longdescription%> wird nicht umgebrochen.
%   Falls das gewünscht ist, \\ mit \renewcommand umschreiben (siehe dazu:
%   http://www.lx-office.org/uploads/media/Lx-Office_Anwendertreffen_LaTeX-Druckvorlagen-31.01.2011_01.pdf)
%
\setlength\LTleft\parindent     % Tabelle beginnt am linken Textrand
\setlength\LTright{0pt}         % Tabelle endet am rechten Textrand
\begin{longtable}{@{}rrp{10.7cm}@{\extracolsep{\fill}}@{}}
% Tabellenkopf
\hline
\textbf{\position} & \textbf{\menge} & \textbf{\bezeichnung} \\
\hline\\
\endhead

% Tabellenkopf erste Seite
\hline
\textbf{\position} & \textbf{\menge} & \textbf{\bezeichnung} \\
\hline\\[-0.5em]
\endfirsthead

% Tabellenende
\\
\multicolumn{3}{@{}r@{}}{\weiteraufnaechsterseite}
\endfoot

% Tabellenende letzte Seite
\hline\\
\endlastfoot

% eigentliche Tabelle
<%foreach number%>
          <%runningnumber%> &
          <%qty%> <%unit%> &
          \textbf{<%description%>} \\*  % kein Umbruch nach der ersten Zeile, damit Beschreibung und Langtext nicht getrennt werden

          <%if longdescription%> && \scriptsize <%longdescription%>\\<%end longdescription%>
          <%if projectnumber%> && \scriptsize \projektnummer: <%projectnumber%>\\<%end projectnumber%>
          <%if reqdate%> && \scriptsize \lieferdatum: <%reqdate%>\\<%end reqdate%>
          <%if serialnumber%> && \scriptsize \seriennummer: <%serialnumber%>\\<%end serialnumber%>
          <%if ean%> && \scriptsize \ean: <%ean%>\\<%end ean%>
          <%foreach si_number%><%if si_number%> && \scriptsize \charge: <%si_chargenumber%> <%if si_bestbefore%> \mhd: <%si_bestbefore%><%end if%> <%si_qty%>~<%si_unit%><%end si_chargenumber%>\\<%end si_number%>

          \\[-0.8em]
<%end number%>

\end{longtable}

\vspace{0.2cm}

<%if notes%>
        \vspace{5mm}
        <%notes%>
        \vspace{5mm}
<%end if%>

<%if delivery_term%>
  \lieferung ~<%delivery_term.description_long%>\\
<%end delivery_term%>

\end{document}

